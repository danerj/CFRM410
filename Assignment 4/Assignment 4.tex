\documentclass[letterpaper,12pt,fleqn]{article}
\usepackage[margin=64pt]{geometry}
\usepackage{amsthm}
\usepackage{amsmath}
\usepackage{amssymb}
\usepackage{parskip}
\usepackage{graphicx}
\usepackage{enumerate}
\usepackage{xcolor}
\usepackage{hyperref}


\newcommand{\transpose}{^{\mbox{\tiny T}}}


\begin{document}
\pagestyle{empty}

\hrule \vspace{0.5em}
\noindent {\bf CFRM 410} \hfill Assignment 4. Name: Dane Johnson. \newline \hrule

\vspace{1em}

Homework policy: you must show your work to receive credit for these exercises.  It is your responsibility to convince the grader that you understand how to solve each of these exercises and to explain precisely how you arrived at your solution.

\vspace{1em}

\begin{enumerate}
\item Suppose we roll two $6$-sided dice, one red and one blue.

\begin{enumerate}[(a)]
\item Describe the sample space $S$. What are the elementary outcomes?

The sample space $S$ for this experiment can be described as the set of ordered pairs $(r, b)$, where $r$ is the result of rolling the red die and $b$ is the result of rolling the blue die (where we arbitrarily fix the red die's outcome to be the first entry in the ordered pair and blue's outcome the second. The elementary outcomes are the $6^2$ possible ordered pairs that may occur. The elementary outcomes are enumerated below.
\begin{align*} S = &\{(1,1), (1,2), (1,3), (1,4), (1,5),        (1,6), \\ & (2,1), (2,2), (2,3), (2,4), (2,5), (2,6), \\ & (3,1), (3,2), (3,3), (3,4), (3,5), (3,6), \\ &
		 (4,1), (4,2), (4,3), (4,4), (4,5), (4,6), \\ &
		 (5,1), (5,2), (5,3), (5,4), (5,5), (5,6), \\ &
		 (6,1), (6,2), (6,3), (6,4), (6,5), (6,6) \}.
\end{align*}


\item What assumption do you need to make to assign probabilities to events defined on $S$?

One assumption we must make to assign probabilities to events defined on $S$ is that each of the elementary outcomes are equally likely. That is, each outcome $(r,b)$ occurs with probability $\frac{1}{n} = \frac{1}{36}$.
\end{enumerate}

\vspace{1em}

Next, consider the following events:

\begin{itemize}
\item[] $A :=$ the value of the red die is even,
\item[] $B :=$ the value of the blue die is odd,
\item[] $C :=$ the sum of the two values is 6, and 
\item[] $D :=$ the sum of the two values is 7.
\end{itemize}

Use the definition of independence of events to answer the remaining parts.

\vspace{0.5em}

\begin{enumerate}[(a)]
\setcounter{enumii}{2}
\item Are the events $A$ and $B$ independent?

Yes. The result of rolling one die cannot affect the other.

\item Are the events $A$ and $C$ independent?

No. $P(A \cap C) \neq P(A)P(C)$.

$$P(A \cap C) = P(\{(2,4), (4,2)\}) = \frac{2}{36},$$
$$P(A)P(C) = \frac{1}{2} \frac{5}{36} = \frac{5}{72} \neq \frac{2}{36}.$$

\item Are the events $A$ and $D$ independent?

Yes they are independent since $P(A \cap C) = P(A)P(C)$.
$$P(A \cap D) = P(\{(2,5), (4,3), (6,1)\}) = \frac{3}{36},$$
$$P(A)P(D) = \frac{1}{2} \frac{6}{36}  = \frac{3}{36} = P(A \cap D).$$

\end{enumerate}


\vspace{2em}

\item Everyone knows that in Seattle it rains $4$ days out of every $10$ days.  I bring my umbrella with me $4$ out of every $5$ days that it rains but only on $1$ out of every $3$ days when it doesn't.

\begin{enumerate}[(a)]
\item What is the probability that I will bring my umbrella with me tomorrow?

Assuming you did not check the weather and change your knowledge of what the weather will be, letting $U$ be the event that you bring your umbrella, and $R$ be the event that it is rainy:

$$P(U) = P(U|R)P(R) + P(U|R^C)P(R^C) = \frac{4}{5}\frac{4}{10} + \frac{1}{3}\frac{6}{10} = \frac{13}{25} = 0.52.$$

\item Suppose that I have my umbrella with me tomorrow.  What is the probability that it is raining? Express your answer as a fraction.

Using the rule for events $A$, $B$ given by $P(A \cap B) = P(A|B)P(B)$,

$$P(R|U) = \frac{P(R \cap U)}{P(U)}= \frac{P(U \cap R)}{P(U)} = \frac{P(U|R)P(R)}{P(U)}= \frac{\frac{4}{5}\frac{4}{10}}{\frac{13}{25}} = \frac{8}{13}.$$
\end{enumerate}





\end{enumerate}



\end{document}